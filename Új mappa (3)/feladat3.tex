\documentclass[12pt]{article}
\usepackage[utf8]{inputenc}
\usepackage{graphicx}
\usepackage{caption}
\usepackage{subcaption}
\usepackage{array}
\usepackage{multirow}
\usepackage{wrapfig}
\usepackage{float}
\usepackage{listings}
\usepackage{xcolor}
\usepackage{colortbl}
\usepackage{hulipsum}
\usepackage{hyperref}

% Csomagok a táblázatokhoz
\usepackage{longtable}
\usepackage{algpseudocode}
\usepackage{algorithm}

% Színek definiálása a listings csomaghoz
\lstset{
  basicstyle=\ttfamily,
  keywordstyle=\color{red},
  commentstyle=\color{gray},
  stringstyle=\color{blue},
  tabsize=2,
  frame=single,
  breaklines=true,
  numbers=left,
  numbersep=10pt,
  numberstyle=\tiny\color{black},
}

% Új float környezetek Python és C kódhoz
\floatstyle{ruled}
\newfloat{PythonCode}{H}{pyc}[section]
\floatname{PythonCode}{Python kód}

\newfloat{CCode}{H}{cc}[section]
\floatname{CCode}{C kód}

\begin{document}

\title{Speciális szoftverek - Ábrák, Táblázatok és Programkódok}
\author{Vadon Viktória}
\date{2023/24/I. félév}
\maketitle

\tableofcontents
\listoffigures
\listoftables

\section{Ábrák}

\subsection{Kép beillesztése szövegbe}
Először egy kép beillesztése szöveg közé torzítás nélkül. Az alábbiakban látható egy példa:

\hulipsum[1]

\begin{wrapfigure}{l}{5cm}
    \includegraphics[width=5cm,height=5cm,keepaspectratio]{example-image}
\end{wrapfigure}

\hulipsum[2]

\subsection{Úszó ábra}
Most egy úszó ábrát illesztünk be.

\begin{figure}[h]
    \centering
    \includegraphics[width=5cm]{example-image}
    \caption{Példa kép úszó ábrában}
    \label{fig:float1}
\end{figure}

Megfigyelhetjük, hogy az ábra megfelelően van elhelyezve.

\subsection{Transzformált kép}
Most beillesztünk egy transzformált képet az előző ábra környezetbe:

\begin{figure}[h]
    \centering
    \includegraphics[width=5cm,angle=90]{example-image}
    \caption{Tükrözött kép}
    \label{fig:transformed}
\end{figure}

Mindkét ábrának saját felirata van, és megfigyelhetjük, hogy több felirat is megadható.

\subsection{Részábrák}
Készítünk egy részábra környezetet, amelyben két képet helyezünk el egymás mellett:

\begin{figure}[h]
    \centering
    \begin{subfigure}[b]{0.4\textwidth}
        \includegraphics[width=\textwidth]{example-image}
        \caption{Első kép}
        \label{fig:sub1}
    \end{subfigure}
    \hspace{1cm}
    \begin{subfigure}[b]{0.4\textwidth}
        \includegraphics[width=\textwidth, angle=90]{example-image}
        \caption{Második kép}
        \label{fig:sub2}
    \end{subfigure}
    \caption{Részábrák elrendezése}
    \label{fig:main_figure}
\end{figure}

\section{Táblázatok}

\subsection{Egyszerű táblázat}
Az első táblázatot az alábbiak szerint készítjük el:

\begin{table}[h]
    \centering
    \begin{tabular}{|p{30pt}|l|c|r|}
        \hline
        \textbf{Oszlop 1} & \textbf{Balra} & \textbf{Középre} & \textbf{Jobbra} \\
        \hline
        Cell 1 & Cell 2 & Cell 3 & Cell 4 \\
        \hline
        Cell 5 & Cell 6 & Cell 7 & \\
        \hline
        \multirow{2}{*}{Cell 8} & Cell 9 & Cell 10 & Cell 11 \\
        \cline{2-4}
        & Cell 12 & Cell 13 & Cell 14 \\
        \hline
    \end{tabular}
    \caption{Első táblázat}
    \label{tab:table1}
\end{table}

\subsection{Színes táblázatok}
Az alábbi táblázat sorainak színezését mutatja be:

\begin{table}[h]
    \centering
    \rowcolors{2}{gray!25}{white}
    \begin{tabular}{|l|c|r|}
        \hline
        Balra & Középre & Jobbra \\
        \hline
        Tartalom 1 & Tartalom 2 & Tartalom 3 \\
        Tartalom 4 & Tartalom 5 & Tartalom 6 \\
        Tartalom 7 & Tartalom 8 & Tartalom 9 \\
        \hline
    \end{tabular}
    \caption{Váltakozó színes sorokkal}
    \label{tab:color_table}
\end{table}

\section{Verbatim}

\subsection{Inline verbatim példa}
Egy példa inline \texttt{verbatim} használatra: Az \verb|\textbf| parancs vastag betűt eredményez.

\subsection{Verbatim környezet példa}

\begin{verbatim}
\begin{enumerate}
    \item Első elem
    \item Második elem
    \item Harmadik elem
\end{enumerate}
\end{verbatim}

\section{Programkód 1: Python}

\begin{PythonCode}
    \caption{Példa Python kód}
    \label{lst:python_example}
    \lstinputlisting[language=Python]{example.py}
\end{PythonCode}

\section{Programkód 2: C}

\begin{CCode}
    \caption{Példa C kód függvénnyel}
    \label{lst:c_example}
    \lstinputlisting[language=C]{example.c}
\end{CCode}

\end{document}
