\documentclass[a4paper,12pt]{article}
% Szükséges csomagok
\usepackage{amsthm}
\usepackage{amsmath, mathtools}
\usepackage{amsfonts, amssymb}
\usepackage{hyperref}
\usepackage{thmtools}

% a) Tételkörnyezet definiálása
\newtheorem{theorem}{Tétel}
\newtheorem{lemma}[theorem]{Lemma}

% d) Definíció környezet definiálása, section-önként számozva
\newtheorem{definition}{Definíció}[section]

\begin{document}

\title{Matematikai Tételek és Formulák}
\author{Feladatmegoldás}
\date{\today}

\maketitle

\tableofcontents

\section{Bevezetés a Tételkörnyezetekbe}

% Első tétel
\begin{theorem}
Ez az első tételünk, ami demonstrálja a tételkörnyezet használatát.
\end{theorem}

% Második tétel névvel/szerzővel
\begin{theorem}[Példa Tétel, Szerző: Gauss]
Ez a második tétel, névvel és szerzővel ellátva.
\end{theorem}

% Lemma
\begin{lemma}\label{lemma:1}
Ez egy lemma, amely a tételekkel együtt van sorszámozva.
\end{lemma}

% Tétel bizonyítása
\begin{proof}
Ez a tétel bizonyítása, ami a tételhez tartozik.
\end{proof}

% Lemma bizonyítása, kereszthivatkozás a Lemmára
\begin{proof}[A Lemma bizonyítása]
Ez a bizonyítás a \ref{lemma:1}. Lemmára hivatkozik.
\end{proof}

\section{Matematikai Definíciók}

% Első definíció
\begin{definition}
Ez az első definíció az első szakaszban.
\end{definition}

% Második definíció
\begin{definition}
Ez a második definíció az első szakaszban.
\end{definition}

\section{További Matematikai Definíciók}

% Harmadik definíció a második szakaszban
\begin{definition}
Ez egy definíció a második szakaszban.
\end{definition}

\section{Matematikai Formulák A-tól Z-ig}

% a) Például egy inline formula
Az \( a^2 + b^2 = c^2 \) egy Pitagorasz-tételként ismert formula.

% b) Többsoros formulák és igazítások
\begin{align}
    f(x) &= x^2 + 2x + 1 \\
    g(x) &= x^2 - 3x + 2
\end{align}

% c) Sigma szimbólum használata
A \(\sigma\) szórás jelentésében fontos szerepet játszik a statisztikában.

% d) Kézírásos nagy I betű használata
Az \(\mathcal{I}\) az intervallum jelölésére szolgál.

\section{További Matematikai Formulák}

% a) Az 1/n^2 sorösszege
Az \( \frac{1}{n^2} \) sorösszege a következőképpen alakul:
\[
\sum_{n=1}^{\infty} \frac{1}{n^2} = \frac{\pi^2}{6}.
\]

% b) Faktoriális definíció
Az \( n! \) (n faktoriális) a számok szorzata 1-től \( n \)-ig, azaz:
\[
n! := \prod_{k=1}^{n} k = 1 \cdot 2 \cdot \dots \cdot n. \tag{1}
\]
Konvenció szerint \( 0! = 1 \).

% c) Binomiális együttható
Legyen \( 0 \leq k \leq n \). A binomiális együttható a következőképpen definiálható:
\[
\binom{n}{k} := \frac{n!}{k! \cdot (n - k)!},
\]
ahol a faktoriálist az (1) szerint definiáljuk.

% d) Szignum függvény
Az előjel- vagy szignumfüggvényt a következőképpen definiáljuk:
\[
\text{sgn}(x) :=
\begin{cases}
    1, & \text{ha } x > 0, \\
    0, & \text{ha } x = 0, \\
   -1, & \text{ha } x < 0.
\end{cases}
\]

\section{Logikai Azonosságok}

Tekintsük az \( L = \{0, 1\} \) halmazt, ahol \( a, b, c, d \in L \). Az alábbiakban belátjuk a következő logikai azonosságot:

\[
(a \land b) \land (c \land d) = a \land (b \land (c \land d)).
\]

Ez az azonosság az asszociativitás törvényére épül. A következő táblázatban bemutatjuk a logikai műveleteket.

\subsection{Logikai Műveletek Táblázata}

A logikai műveletek táblázata a következőképpen alakul:

\begin{table}[h]
\centering
\begin{tabular}{|c|c|c|c|}
\hline
\( a \) & \( b \) & \( a \land b \) & \( a \lor b \) \\ \hline
0 & 0 & 0 & 0 \\ \hline
0 & 1 & 0 & 1 \\ \hline
1 & 0 & 0 & 1 \\ \hline
1 & 1 & 1 & 1 \\ \hline
\end{tabular}
\caption{Logikai műveletek táblázata: AND (\(\land\)) és OR (\(\lor\))}
\end{table}

\subsection{Az azonosság levezetése}

1. **Alapállítás**:
   \[
   (a \land b) \land (c \land d).
   \]

2. **Alkalmazzuk az asszociatív törvényt**:
   \[
   = a \land (b \land (c \land d)).
   \]

3. **Megmutatjuk a kommutativitás törvényét**:
   \[
   a \land b = b \land a \quad \text{és} \quad a \lor b = b \lor a.
   \]

4. **Alkalmazzuk a kommutativitás törvényét**:
   \[
   a \land (b \land c) = (a \land b) \land c.
   \]

5. **Végül kifejtjük**:
   \[
   (a \land b) \land (c \land d) = a \land (b \land (c \land d)).
   \]

\subsection{Összegzés}

A fenti levezetés alapján beláttuk, hogy:

\[
(a \land b) \land (c \land d) = a \land (b \land (c \land d)).
\]

Ez a logikai azonosság az asszociatív törvényre épít, amely lehetővé teszi a kifejezések átcsoportosítását. A kommutativitás törvénye tovább erősíti ezt az azonosságot.

\end{document}
