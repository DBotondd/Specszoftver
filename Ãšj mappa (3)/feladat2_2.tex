\documentclass[12pt]{article}
\usepackage[utf8]{inputenc}
\usepackage[twocolumn]{geometry} % Kéthasábos beállítás
\usepackage{fancyhdr} % Fejléc, lábléc kezelése
\usepackage[inline]{enumitem} % Egysoros listákhoz
\usepackage{babel} % Magyar nyelv támogatása
\usepackage{hyperref} % Hivatkozások
\usepackage{xcolor} % Színek használata
\usepackage{hulipsum} % Magyar helykitöltő szöveg

% Fejléc és lábléc beállítások
\pagestyle{fancy}
\fancyhf{}
\fancyhead[LE]{\thepage} % Páros oldalon oldalszám balra
\fancyhead[RO]{\thepage} % Páratlan oldalon oldalszám jobbra
\fancyhead[LO]{\leftmark} % Páratlan oldalon section neve
\fancyhead[RE]{\rightmark} % Páros oldalon subsection neve
\fancyfoot[C]{Egyetem Neve} % Lábjegyzet közepén egyetem neve
\renewcommand{\footrulewidth}{0.4pt} % Lábjegyzet feletti vonal

% Első oldal plain stílus módosítása
\fancypagestyle{plain}{
  \fancyhf{}
  \fancyfoot[LE,RO]{\thepage} % Oldalszám láblécben
  \renewcommand{\footrulewidth}{0.4pt} % Vonal a lábléc felett
}

% Fejléc magasság beállítása
\setlength{\headheight}{15pt}

\begin{document}

\title{LaTeX Dokumentum Feladatok}
\author{Saját Név}
\date{\today}
\maketitle
\thispagestyle{plain} % Első oldal speciális stílus

\section{Fejléc és Lábléc Beállítások}
Ezen a dokumentumon a kéthasábos és egyhasábos szerkesztést, valamint a különböző oldalstílusok használatát gyakoroljuk.

\subsection{Kéthasábos Dokumentum}
Az első feladat egy kéthasábos dokumentumban történik. A fejléc páros és páratlan oldalakon eltérő szöveget tartalmaz, például saját nevünket és az egyetem nevét.

\hulipsum[1]

\subsection{Fejléc és Lábléc Testreszabása}
A lábléc közepén az egyetem neve szerepel, míg a lábléc fölött egy választóvonal jelenik meg. A lábléc külső sarkában az oldalszám található.

\hulipsum[2]

\section{Egysoros Lista}
Az `enumitem` csomaggal létrehozhatunk egysoros listákat. Az alábbi lista például három elemű, az utolsó elem elé pedig automatikusan beillesztjük az „és” szót:

\begin{itemize}[label=---]
  \item Első elem
  \item Második elem
  \item[és] Harmadik elem
\end{itemize}

\section{Számozott Lista}
Készítsünk beágyazott számozott listákat, és figyeljük meg a különböző szinteken alkalmazott számozási stílusokat. Az alábbi példa mutatja, hogy hogyan mélyíthetjük a beágyazásokat:

\begin{enumerate}
  \item Első szint
  \begin{enumerate}
    \item Második szint
    \begin{enumerate}
      \item Harmadik szint
      \begin{enumerate}
        \item Negyedik szint
        \begin{enumerate}
          \item Ötödik szint
        \end{enumerate}
      \end{enumerate}
    \end{enumerate}
  \end{enumerate}
\end{enumerate}

\subsection{Egyedi Lista Típus}
Az `enumerate` klónozásával létrehozhatunk egy saját listatípust, mely plusz egy beágyazási szintet tesz lehetővé. Az alapértelmezett számozás minden szinten zárójelbe tett arab számokkal történik:

\begin{enumerate}[label=(\arabic*)]
  \item Első elem
  \begin{enumerate}[label=(\arabic*)]
    \item Második elem
    \begin{enumerate}[label=(\arabic*)]
      \item Harmadik elem
    \end{enumerate}
  \end{enumerate}
\end{enumerate}

\subsection{Folytatott Lista}
Az alábbi lista folytatja az előző számozott lista számozását:

\begin{enumerate}[start=4]
  \item Negyedik elem
  \item Ötödik elem
\end{enumerate}

\section{Leíró Lista}
Készítsünk egy leíró listát három elemmel. Az első elemnek nincs címkéje, a második elem rövid, míg a harmadik elem hosszú címkét kap:

\begin{description}
  \item[] Nincs címke: \hulipsum[3]
  \item[Kis címke] Rövid címke: \hulipsum[4]
  \item[Egy nagyon hosszú címke, ami egy sornál is hosszabb] Hosszú címke: \hulipsum[5]
\end{description}

A szócímke legyen dőlt (`slanted`):

\begin{description}
  \item[\textsl{Dőlt címke}] \hulipsum[6]
\end{description}

\end{document}
