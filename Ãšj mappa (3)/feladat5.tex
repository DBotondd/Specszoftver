\documentclass[12pt,16:9]{beamer}
\usepackage{hyperref}
\usepackage{xcolor}
\usepackage{multimedia}

% Set default theme
\usetheme{Madrid}
\usecolortheme{beetle}

\title{Beamer Prezentáció}
\author{Név}
\date{\today}

\begin{document}

% Cím dia
\begin{frame}
    \titlepage
\end{frame}

% Tartalomjegyzék dia
\begin{frame}
    \frametitle{Tartalomjegyzék}
    \tableofcontents
\end{frame}

% 1. Feladat (documentclass, frame)
\section{Documentclass és Frame}

\subsection{Alapbeállítások}
\begin{frame}
    \frametitle{Első Frame}
    \framesubtitle{A Beamer dokumentum bevezetése}
    Ez az első frame, amely bemutatja a Beamer osztály használatát!
\end{frame}

\begin{frame}
    \frametitle{Dummy Frame 1}
    Ez egy dummy frame, aminek más a címe.
\end{frame}

\begin{frame}
    \frametitle{Dummy Frame 2}
    Ez egy másik dummy frame.
\end{frame}

% Verbatim használata
\begin{frame}[fragile] % fragile option allows verbatim
    \frametitle{Verbatim Példa}
    \begin{verbatim}
    Ez egy verbatim példa.
    Itt bármilyen LaTeX kód szerepelhet!
    \end{verbatim}
\end{frame}

% Több bekezdés
\begin{frame}[allowframebreaks]
    \frametitle{Több Frame}
    Ez a frame több bekezdést tartalmaz.

    Itt van az első bekezdés.
    
    \framebreak
    
    Itt van a második bekezdés, ami automatikusan széttöri a frame-eket.
\end{frame}

% 2. Feladat (Új elemek)
\section{Új Elemenek}

\subsection{Kétoszlopos dia}
\begin{frame}
    \frametitle{Kétoszlopos Dia}
    \begin{columns}
        \begin{column}{0.5\textwidth}
            \begin{itemize}
                \item Első elem
                \item Második elem
            \end{itemize}
            \begin{enumerate}
                \item Első számozott elem
                \item Második számozott elem
            \end{enumerate}
        \end{column}
        \begin{column}{0.5\textwidth}
            \includegraphics[width=\linewidth]{example-image-a} % replace with your image
            \caption{Példa kép}
        \end{column}
    \end{columns}
\end{frame}

\begin{frame}
    \frametitle{Block Példák}
    \begin{block}{Blokk}
        Ez egy blokk példa.
    \end{block}

    \begin{exampleblock}{Példa Blokk}
        Ez egy példa blokk.
    \end{exampleblock}

    \begin{alertblock}{Figyelem Blokk}
        Ez egy figyelmeztető blokk!
    \end{alertblock}

    \begin{block}{} % Cím nélküli blokk
        Ez egy cím nélküli blokk példa.
    \end{block}
\end{frame}

\begin{frame}
    \frametitle{Tétel és Bizonyítás}
    \begin{theorem}
        Ez egy tétel példája.
    \end{theorem}
    \begin{proof}
        Ez a bizonyítás.
    \end{proof}
\end{frame}

\begin{frame}
    \frametitle{Semiverbatim Példa}
    \begin{semiverbatim}
        \textbf{Kiemelt szöveg} és \textcolor{red}{színes szöveg}.
    \end{semiverbatim}
\end{frame}

% 3. Feladat (Szakaszok, tartalomjegyzék)
\section{Szakaszok}

\begin{frame}
    \frametitle{Szakasz Cím Dia}
    \framesubtitle{Első Szakasz}
    Itt bemutatjuk az első szakaszt.
\end{frame}

\begin{frame}
    \frametitle{Tartalomjegyzék}
    \tableofcontents[currentsection]
\end{frame}

\subsection{Alszakaszok}
\begin{frame}
    \frametitle{Első Alszakasz}
    Ez az első alszakasz.
\end{frame}

\begin{frame}
    \frametitle{Második Alszakasz}
    Ez a második alszakasz.
\end{frame}

% 4. Feladat (Témák)
\section{Témák}

\begin{frame}
    \frametitle{Témák Tesztelése}
    Itt teszteljük a Beamer téma matrix-ot!
\end{frame}

% 5. Feladat (Overlay)
\begin{frame}
    \frametitle{Overlay Példa}
    Első sor \pause
    Második sor \pause
    Harmadik sor
\end{frame}

\begin{frame}
    \frametitle{Listák Tesztelése}
    \begin{itemize}
        \item<1-> Első elem
        \item<2-> Második elem
        \item<3-> Harmadik elem
    \end{itemize}
\end{frame}

\begin{frame}
    \frametitle{Tétel + Bizonyítás}
    \begin{theorem}
        Ez a tétel.
    \end{theorem}
    \only<2>{
    \begin{proof}
        Itt a bizonyítás.
    \end{proof}
    }
\end{frame}

\begin{frame}
    \frametitle{Képek váltakozása}
    \only<1>{\includegraphics[width=\linewidth]{example-image-a}} % replace with your image
    \only<2>{\includegraphics[width=\linewidth]{example-image-b}} % replace with your image
\end{frame}

% 6. Feladat (Áttűnések)
\begin{frame}
    \frametitle{Áttűnések Tesztelése}
    \transfade<1-2>
    Ez egy átmenet teszt.
\end{frame}

\begin{frame}
    \frametitle{Automatikus Léptetés}
    \begin{itemize}
        \item<1-> Első elem
        \item<2-> Második elem
        \item<3-> Harmadik elem
    \end{itemize}
\end{frame}

\end{document}
